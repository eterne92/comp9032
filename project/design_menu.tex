\documentclass[a4paper, 12 pt]{report}
\usepackage{multirow}
\usepackage{graphicx}
\usepackage{tabularx}
\usepackage{geometry}
\geometry{left=2.5cm,right=2.5cm,top=2.5cm,bottom=2.5cm}
\usepackage{algorithm}  
\usepackage{tikz}
\usetikzlibrary{shapes.geometric, arrows}



% -------------------------------------------------------------------------------------
% BEGIN DOCUMENT
% -------------------------------------------------------------------------------------
\begin{document}
\title{Drone Design Manual}
\author{Shao Hui z5155945}
\date{}
\maketitle
\pagestyle{empty}
\setcounter{section}{0}
% -------------------------------------------------------------------------------------
% TABLE OF CONTENTS
% -------------------------------------------------------------------------------------
\tableofcontents
\newpage

% -------------------------------------------------------------------------------------
% INTRODUCTION
% -------------------------------------------------------------------------------------
\section{Introduction}

\newpage

% -------------------------------------------------------------------------------------
%-------------------------
\tikzstyle{startstop} = [rectangle, rounded corners, minimum width=3cm, minimum height=1cm,text centered, draw=black, fill=red!30]
\tikzstyle{io} = [trapezium, trapezium left angle=70, trapezium right angle=110, minimum width=3cm, minimum height=1cm, text centered, draw=black, fill=blue!30]
\tikzstyle{process} = [rectangle, minimum width=3cm, minimum height=1cm, text centered, draw=black, fill=orange!30]
\tikzstyle{decision} = [diamond, minimum width=3cm, minimum height=1cm, text centered, draw=black, fill=green!30]
\tikzstyle{arrow} = [thick,->,>=stealth]
% -------------------------------------------------------------------------------------
\section{Diagram}
\begin{center}
	\begin{tikzpicture}[node distance = 2.5cm]
		\node (start) [startstop] {start};
		\node (inputX) [io, below of = start] {input X};
		\node (checkX) [decision, right of = inputX, xshift = 2cm] {check X};
		\node (inputY) [io, below of = inputX] {input Y};
		\node (checkY) [decision, right of = inputY, xshift = 2cm] {check Y};
		
		\draw [arrow] (checkX) |- (start);
	\end{tikzpicture}
\end{center}


\newpage

% -------------------------------------------------------------------------------------
% Mountain Generator
% -------------------------------------------------------------------------------------
\section{Mountain Generator}
This project provided a random mountain generator written in python called generator.py that will generate a 64$\times$64 matrix and write it to the "mountain.asm" file which will be included into the source code.\\\\
This generator uses such technique,\\
1. Randomly generate a peak in position x, y with height h.\\
2. Generate down from the peak to its neighbour grid until height became less then 10.\\
3. Combine this graph with the former mountain map and cut out the positions that are lower than the former map.\\
4. Repeat this procedure for several times.\\\\
This technique always generate a mountain that make sense. You can also use your own generator or simply type in a mountain that suits your need. But make sure when you assemble the project, label your mountain with a label "mountain".
\newpage
% -------------------------------------------------------------------------------------
% Control Procedure
% -------------------------------------------------------------------------------------
\section{Control Procedure}
Every time when the board is reset. The LCD will display a string "Input X:".\\
You should always follow such emulate procedure.\\
1. Input some number within 64 using keypad end with "\#", which is the row number of the accident scene. If your input is invalid, led will flash three times and board will be reset.\\
2. After LCD showed "Input Y:", input some number within 64 using keypad end with "\#", which is the column number of the accident scene. If your input is invalid, led will flash three times and board will be reset.\\
3. After LCD showed "READY", press PB0 to start search. LED will flash 3 times to indicate the start of search. LCD will show drone's current position and flying direction. Motor will be spinning, fast means flying high, low speed means suspending.\\
4. During search, you can always press PB1 to abort the search. LCD will show "ABORT".\\
5. After finding the accident scene. LCD will show "FOUND" and the position of the accident scene. Motor will stop. If a accident scene not fount(which actually never happen during a simulation), LCD will show "NOT FOUND", and motor also stop.
\end{document}
